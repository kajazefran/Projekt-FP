\documentclass{article}

\usepackage{amsmath}
\usepackage{amsfonts} 
\usepackage{amssymb}
\usepackage{mathrsfs}
\usepackage{geometry}
\usepackage{dsfont}

\geometry{a4paper, top=2.5cm, bottom=2.5cm, left=2.0cm, right=2.0cm}
\newtheorem{definition}{Definicija}[section] 

\title{	
    \normalfont\normalsize
	\textsc{Fakulteta za matematiko in fiziko\\ Univerza v Ljubljani}\\ 
	\vspace{25pt}
	\rule{\linewidth}{0.5pt}\\
	\vspace{20pt}
	{\huge \textbf{Splošni Somborjev indeks}}\\ 
	\vspace{12pt}
	\rule{\linewidth}{2pt}\\
	\vspace{12pt}
}

\author{Pecoraro Eliza Katarina, Žefran Kaja}

\date{November 2025} 

\begin{document}

\maketitle

\newpage
\section{Uvod}
Naj bo \( G = (V, E)\) graf sestavljen iz množice vozlišč \( E(G) \) in množice povezav \( V(G) \), kjer je
\( | E(G) \vert = n \) število vozlišč in \( | V(G) \vert = m \) število povezav.
Stopnja vozlišča \( d_G(v_i) \) za neko vozlišče \( v_i \in E(G)\) predstavlja število vozlišč s katerimi 
si je to sosednje, torej med njimi obstaja povezava. Največjo stopnjo vozlišča grafa
\( G \) označimo z \( \Delta (G) \) oziroma, kot v nadaljevanju tudi le kot \( \Delta \).\\

V kemijski matematiki imajo zelo pomembno vlogo topološki indeksi, ki temeljijo na stopnjah 
vozlišč. Ti so uporabljeni v analizah in pri določanju napovedi. Gre za numerične vrednosti, 
ki jih dodelimo molekulam oziroma njihovim strukturnim predstavitvam (grafom).
Eden izmed teh topoloških indeksov je \textit{splošen Somborjev indeks}, ki ga bomo bolj podrobno 
obravnavali pri tem projektnem delu.

\begin{definition} 
	Somborjev indeks \( SO_{\alpha}(G) \) grafa \( G \) definiramo kot:
	\[
	SO_{\alpha}(G) = \sum_{v_i, v_j \in E(G)} (d_G(v_i)^2 + d_G(v_j)^2)^{\alpha}
	\]
	kjer \( d_G(v_i) \in V(G)\) predstavlja stopnjo vozlišča \( v_i \in E(G)\), \( \alpha \) pa je
	poljuben realen parameter.	
\end{definition}

\section{Cilj projekta}
Graf reda \( n \), ki ne vsebuje ciklov imenujemo drevo \( n \)-vozlišč. Zanj uporabimo oznako \( T \in \tau (n, \Delta ) \).\\

V tem projektnem delu želimo indentificirati drevesa z največjo vrednostjo Somborjevega indeksa znotraj razreda 
dreves \( n \)-vozlišč \( \tau (n, \Delta ) \), največje stopnje \( \Delta \) in parametrom \( \alpha \in (0,1) \).\\

\section{Strukura dela}
Pri manjših grafih oziroma drevesih bomo te iskali sistematično, za večje se bomo tega lotili stohastično.
Ločeno bomo preverili kaj se dogaja pri različnih vrednostih \( \Delta \) in \( \alpha \in (0, 1) \).

\end{document}

