\documentclass{article}

\usepackage[slovene]{babel}
\usepackage{amsmath}
\usepackage{amsfonts} 
\usepackage{amssymb}
\usepackage{mathrsfs}
\usepackage{geometry}
\usepackage{dsfont}
\usepackage{graphicx}
\usepackage{float}
\usepackage{tikz}
\usepackage[T1]{fontenc}
\usepackage[utf8]{inputenc}
\usepackage{comment}

\graphicspath{{../slike/}}
\geometry{a4paper, top=2.5cm, bottom=2.5cm, left=2.0cm, right=2.0cm}
\newtheorem{definition}{Definicija}[section] 

\begin{document}

\thispagestyle{empty}

\begin{center}
    \normalfont\normalsize
    \textsc{Fakulteta za matematiko in fiziko\\Univerza v Ljubljani}\\[25pt]
    \rule{\linewidth}{0.5pt}\\[20pt]
    {\huge \textbf{Splošni Somborjev indeks}}\\[12pt]
    \rule{\linewidth}{2pt}\\[18pt]
    {\large \textsc{Poročilo projekta}}
\end{center}

\vfill
\begin{center}
    \large Pecoraro Eliza Katarina, Žefran Kaja\\[18pt]
    \normalsize december 2025
\end{center}

\newpage

\section{Teoretični uvod}

Naj bo \( G = (V, E)\) graf, sestavljen iz množice vozlišč \( V(G) \) in množice povezav \( E(G) \), kjer je
\( | V(G) | = n \) število vozlišč in \( | E(G) | = m \) število povezav.
Stopnja vozlišča \( d_G(v) \) predstavlja število povezav, ki so incidentne z vozliščem \( v \).
Največjo stopnjo grafa \( G \) označimo z \( \Delta(G) \) oziroma, kadar ni dvoumnosti, le z \( \Delta \).

\medskip

Graf reda \( n \), ki ne vsebuje ciklov, imenujemo drevo z \( n \) vozlišči.
Razred vseh dreves reda \( n \) z največjo stopnjo največ \( \Delta \) označimo z \( \tau(n,\Delta) \).

\medskip

V kemijski matematiki imajo pomembno vlogo topološki indeksi, ki temeljijo na stopnjah vozlišč. Gre za numerične
invariante grafov, ki se uporabljajo pri modeliranju molekulskih struktur in napovedovanju njihovih lastnosti.
Eden izmed takšnih indeksov je \textbf{splošni Somborjev indeks}.

\begin{definition}
	Splošni Somborjev indeks \( SO_{\alpha}(G) \) grafa \( G \) definiramo kot
	\[
	SO_{\alpha}(G) = \sum_{\{u,v\} \in E(G)} \bigl(d_G(u)^2 + d_G(v)^2\bigr)^{\alpha},
	\]
	kjer je \( \alpha \in \mathbb{R} \) realen parameter.
\end{definition}

Za \( \alpha = 0 \) je \( SO_{0}(G) = |E(G)| \). Če je \( T \) drevo z \( n \) vozlišči, velja
\( SO_{0}(T) = n - 1 \). V nadaljevanju se bomo osredotočili na primer
\( \alpha \in (0,1) \).

\section{Cilj projekta}

V tem delu raziskave se omejimo na \textbf{drevesa}, saj ta predstavljajo osnovni, a hkrati dovolj raznolik razred grafov,
pri katerem je mogoč popoln pregled vseh neizomorfnih struktur za manjše vrednosti \( n \).

\medskip

Cilj projekta je identificirati drevesa z največjo vrednostjo splošnega Somborjevega indeksa znotraj razreda
\( \tau(n,\Delta) \) za izbran parameter \( \alpha \in (0,1) \).

\medskip

Posebej nas zanima vpliv:
\begin{itemize}
	\item največje stopnje \( \Delta \),
	\item razporeditve stopenj v drevesu,
	\item izbire parametra \( \alpha \),
\end{itemize}
na vrednost indeksa in obliko ekstremnih dreves.

\subsection{Vedenje indeksa pri različnih vrednostih parametra \( \alpha \)}

Čeprav se v projektu osredotočamo na interval \( \alpha \in (0,1) \), nam vpogled v obnašanje indeksa pri drugih
vrednostih parametra pomaga pri razumevanju ekstremnih struktur.

\begin{itemize}
	\item Za \( \alpha > 1 \) imajo večje stopnje vozlišč izrazit vpliv na vrednost indeksa, zato so ekstremna
	drevesa tista z zelo neenakomerno porazdelitvijo stopenj, kot je zvezdno drevo.
	\item Za \( \alpha < 0 \) velike stopnje indeks zmanjšujejo, zato so ugodna drevesa z enakomerno
	porazdeljenimi \mbox{stopnjami}, na primer potna drevesa.
\end{itemize}

Za \( \alpha \in (0,1) \) je funkcija \( x \mapsto x^{\alpha} \) naraščajoča in konkavna, kar pomeni, da
večje stopnje še vedno povečujejo vrednost indeksa, vendar z zmanjšanim prispevkom.

\newpage

\section{Analiza dreves}

V tem poglavju analiziramo obnašanje splošnega Somborjevega indeksa pri različnih vrednostih
parametra $\alpha$ na drevesih različnih velikosti. Z analizo želimo raziskati
vpliv parametra $\alpha \in (0,1)$ na vrednost splošnega Somborjevega indeksa ter preučiti vpliv
stopnje razvejanosti in števila vozlišč.

\medskip

V nadaljevanju najprej obravnavamo majhne grafe, kjer je zaradi omejenega števila
vozlišč mogoč izčrpen računski pregled vseh neizomorfnih dreves, nato pa analizo
razširimo še na večje grafe.

\subsection{Računski pristop in generiranje podatkov}

Analiza splošnega Somborjevega indeksa je bila izvedena z uporabo dveh različnih računskih pristopov, \mbox{odvisno} od velikosti 
obravnavanih dreves. Pri majhnih drevesih smo uporabili izčrpen, sistematičen pregled vseh \mbox{neizomorfnih} struktur, 
pri večjih drevesih pa smo zaradi računske zahtevnosti uporabili stohastično iskanje.

\medskip

\paragraph{Majhna drevesa}
Za majhne vrednosti števila vozlišč (za $n \leq 10$) smo generirali vsa neizomorfna drevesa dane velikosti. 
Za vsako drevo smo izračunali največjo stopnjo $\Delta$ ter vrednost Somborjevega indeksa za izbrane vrednosti parametra
\[
\alpha \in \{0.05, 0.45, 0.5, 0.55, 0.95\}.
\]
ki ležijo znotraj intervala $(0,1)$. Izbrane vrednosti omogočajo opazovanje obnašanja
indeksa v bližini mejnih vrednosti $0$ in $1$ ter v okolici srednje vrednosti $0.5$.
Drevesa smo nato razvrstili glede na največjo stopnjo in identificirali tista, pri katerih Somborjev indeks 
doseže maksimalno vrednost. Ta pristop omogoča natančno primerjavo vseh možnih struktur in jasen vpogled v vpliv 
razvejanosti na vrednost indeksa.

\medskip

\paragraph{Velika drevesa}
Pri večjih vrednostih $n$ izčrpno generiranje vseh dreves ni več izvedljivo, zato smo uporabili stohastični pristop. 
Pri tem smo uporabili iste vrednosti parametra $\alpha$ kot pri majhnih drevesih, da so rezultati med seboj primerljivi. 
Algoritem začne z naključno izbranim začetnim drevesom, ki zadošča danim omejitvam (število vozlišč in največja stopnja), ter nato drevo postopno izboljšuje z lokalnimi spremembami (izbrali smo list - vozlišče stopnje 1, ta
list priključili na drugo vozlišče) katerega stopnja po priključitvi ne preseže $\Delta$. Pri vsakem koraku se iz trenutnega drevesa generira novo drevo z majhno strukturno spremembo, pri čemer se sprememba sprejme le, če ne zmanjša vrednosti Somborjevega indeksa.

\medskip

Ker je število možnih vrednosti največje stopnje $\Delta$ pri velikih drevesih zelo veliko, smo stohastično preizkusili le 
nekatere reprezentativne vrednosti. 
Izbrali smo si: 
\[
    \Delta \in \{2, 3, 4, 5, 10, \left\lfloor \frac{n}{4} \right\rfloor, \left\lfloor \frac{n}{2} \right\rfloor, n - 1 \}.
\]  
Na primer, za drevo z $n=60$ smo opazovali vrednost pri
$
\Delta \in \{2, 3, 4, 5, 10, 15, 30, 59\}.
$

\newpage
\subsection{Psevdokoda algoritma}

Za boljši pregled nad izvedbo analize smo predstavili osnovno logiko algoritma v psevdokodi.

\paragraph{Sistematični pregled za majhna drevesa:}

\begin{verbatim}
Za n = 1 do 10:
    Generiraj vsa neizomorfna drevesa T_n
    Za vsako drevo T:
        Izracunaj najvecjo stopnjo Delta(T)
        Izracunaj SO_alpha(T) za alpha v {0.05, 0.45, 0.5, 0.55, 0.95}
    Identificiraj drevesa z najvecjim SO_alpha
\end{verbatim}

\paragraph{Stohastični pristop za velika drevesa:}

\begin{verbatim}
Stohastični pristop za velika drevesa:
Za dano n in alpha:
    Za več začetnih dreves (n_starts):
        Nastavi drevo T naključno
        Ponovi n_steps:
            Izberi lokalno spremembo T'
            Če SO_alpha(T') > SO_alpha(T):
                Sprejmi spremembo
    Shrani drevo z največjim SO_alpha
\end{verbatim}

\subsection{Analiza majhnih dreves}

Pri analizi majhnih grafov, ki vsebujejo vse do 10 vozlišč, opazimo izrazit in stabilen trend
v obnašanju \mbox{splošnega Somborjevega indeksa}. Za vse uporabljene vrednosti parametra
$\alpha \in (0,1)$ se izkaže, da je vrednost Somborjevega indeksa največja pri grafih z
največjo možno maksimalno stopnjo $\Delta$, glede na število vozlišč, torej pri $\Delta = n-1$ . To pomeni, da imajo
vozlišča z največjim številom povezav oziroma z največjo stopnjo ključno vlogo pri velikosti
indeksa, saj bistveno zaznamujejo celotno strukturo grafa.

\medskip
Pri analizi majhnih grafov smo uporabili vrednosti parametra
\[
\alpha \in \{0.05, 0.45, 0.5, 0.55, 0.95\},
\]
Rezultati so pokazali, da večja kot je vrednost parametra $\alpha$, večja je tudi
vrednost Somborjevega indeksa, ne glede na število vozlišč ali stopnjo razvejanosti grafa.

\medskip

Za vsako fiksno število vozlišč $n$ smo izvedli popoln računski pregled vseh
neizomorfnih dreves. Za vsak graf smo najprej izračunali največjo stopnjo $\Delta$
ter drevesa razvrstili v razrede glede na to vrednost. Znotraj vsakega razreda
smo izračunali vrednost Somborjevega indeksa
\[
SO_\alpha(G) = \sum_{uv \in E(G)} (d_u^2 + d_v^2)^\alpha,
\]
kjer $d_u$ in $d_v$ označujeta stopnji krajišč roba $uv$, ter določili drevo, pri
katerem indeks doseže največjo vrednost za dano $\Delta$.

\medskip

Poleg tega smo za vsako izbrano vrednost parametra $\alpha$ določili tudi globalni
maksimum Somborjevega indeksa med vsemi drevesi z $n$ vozlišči ter identificirali
vrednost največje stopnje $\Delta$, pri kateri se ta maksimum pojavi.

\medskip

Kot reprezentativen primer si oglejmo primer $n = 7$.
Rezultati pokažejo, da Somborjev indeks za vse \mbox{obravnavane} vrednosti parametra $\alpha$
monotono narašča z naraščanjem maksimalne stopnje $\Delta$.
Globalni \mbox{maksimum} je v vseh primerih dosežen pri drevesu z $\Delta = 6$,
kar ustreza zvezdastemu drevesu.
To drevo vsebuje eno vozlišče stopnje $6$, ki je povezano z vsemi preostalimi vozlišči,
kar povzroči največji možni prispevek k vrednosti indeksa.

\medskip

Opazimo tudi, da se z večanjem parametra $\alpha$ razlike med vrednostmi
Somborjevega indeksa pri različnih vrednostih $\Delta$ še povečujejo.
To pomeni, da večje vrednosti $\alpha$ dodatno poudarijo prispevek robov,
ki povezujejo vozlišča z visokimi stopnjami, kar pride posebej do izraza
pri zvezdastih drevesih.

\begin{figure}[H]
\centering
\includegraphics[width=0.25\textwidth]{zvezda7.png}
\caption{Zvezdno drevo reda $n=7$ z maksimalno stopnjo $\Delta = 6$.}
\label{fig:zvezda7}
\end{figure}

\begin{table}[H]
\centering
\begin{tabular}{c|ccccc}
\hline
$\alpha$ 
& $\Delta=2$ 
& $\Delta=3$ 
& $\Delta=4$ 
& $\Delta=5$ 
& $\Delta=6$ \\
\hline
0.05 & 6.61 & 6.76 & 6.87 & 6.98 & 7.19 \\
0.45 & 14.32 & 17.62 & 20.63 & 23.94 & 30.47 \\
0.50 & 15.79 & 19.86 & 23.69 & 28.02 & 36.50 \\
0.55 & 17.40 & 22.39 & 27.22 & 32.80 & 43.72 \\
0.95 & 38.07 & 58.52 & 83.37 & 117.49 & 185.33 \\
\hline
\end{tabular}
\caption{Maksimalne vrednosti splošnega Somborjevega indeksa $SO_\alpha$ za drevesa reda $n=7$ in vse možne maksimalne stopnje $\Delta$.}
\end{table}

\medskip

Na podlagi izčrpne računske analize majhnih grafov lahko sklenemo, da za vsako
obravnavano število vozlišč $n \leq 10$ in za vsak parameter $\alpha \in (0,1)$
Somborjev indeks na razredu dreves doseže največjo vrednost pri grafih z
največjo možno maksimalno stopnjo $\Delta = n-1$. Z večanjem števila vozlišč se vrednost Somborjevega indeksa povečuje,
vendar osnovni trendi, povezani z vplivom parametra $\alpha$ in strukture drevesa,
ostajajo enaki. To omogoča smiselno primerjavo rezultatov med majhnimi in večjimi drevesi.

\subsection{Analiza velikih dreves}

Pri analizi dreves z večjim številom vozlišč se osnovne lastnosti splošnega Somborjevega indeksa v veliki meri ohranijo. 
Absolutne vrednosti indeksa z naraščanjem števila vozlišč pričakovano naraščajo, pri čemer se za vse obravnavane vrednosti parametra
$\alpha \in (0,1)$ globalni maksimum še vedno doseže pri drevesih z največjo možno stopnjo, torej pri zvezdnem drevesu z
$\Delta = n-1$.

\medskip

Kljub temu pa se v primerjavi z majhnimi drevesi pojavijo pomembne strukturne razlike pri drevesih z \mbox{omejeno} največjo stopnjo
$\Delta$. Medtem ko je pri majhnih grafih optimalna struktura za dano $\Delta$ \mbox{pogosto} blizu \mbox{zvezdni} ali močno centralizirani obliki, 
se pri večjih drevesih optimalne konfiguracije porazdelijo na več \mbox{vozlišč} z visoko stopnjo. Empirični rezultati kažejo, da se pri 
naraščanju $\Delta$ število vozlišč, ki \mbox{dosegajo} \mbox{maksimalno} \mbox{stopnjo}, postopoma zmanjšuje, hkrati pa se pojavljajo vozlišča z drugo 
največjo stopnjo, ki \mbox{pomembno} \mbox{prispevajo} k skupni vrednosti indeksa.

\medskip

Ta pojav je povezan z naravo izraza
\[
\left( d_u^2 + d_v^2 \right)^{\alpha},
\]
kjer za $\alpha \in (0,1)$ posamezni zelo veliki prispevki ne prevladajo povsem nad večjim številom zmerno velikih prispevkov. 
Posledično so pri fiksni omejitvi $\Delta$ optimalna drevesa pogosto sestavljena iz več delno razvejanih vozlišč, namesto 
enega samega izrazito dominantnega vozlišča.

\medskip
Analiza za večje vrednosti $n$ tako kaže, da se s povečevanjem največje dovoljene stopnje $\Delta$ vrednost \mbox{Somborjevega} 
indeksa monotono povečuje, vendar se struktura optimalnih dreves postopno spreminja: od dreves z večjim številom vozlišč maksimalne 
stopnje pri manjših $\Delta$ do struktur z enim samim dominantnim \mbox{vozliščem} v primeru $\Delta = n-1$. Ta prehod potrjuje, da je pri 
velikih drevesih vpliv porazdelitve stopenj vozlišč \mbox{ključen} za \mbox{razumevanje} obnašanja splošnega Somborjevega indeksa, čeprav globalni 
maksimum ostaja dosežen pri \mbox{ekstremno} razvejanih strukturah.

\medskip

Pri velikih drevesih se zvezdno drevo ne izkaže kot optimalna struktura. Razlog je v tem, da pri zelo veliki maksimalni 
stopnji $\Delta$ večina robov povezuje vozlišče z zelo visoko stopnjo z listi, kar omeji skupni prispevek ostalih robov. 
Drevesa z bolj uravnoteženo porazdelitvijo stopenj omogočajo, da večje število robov prispeva k vsoti v definiciji 
Somborjevega indeksa, kar je za $\alpha \in (0,1)$ ugodneje.

\medskip

Rezultati stohastičnega iskanja se lahko med posameznimi zagoni nekoliko razlikujejo, kar je pričakovano zaradi 
naključne narave algoritma. Kljub temu se dobljene optimalne strukture dosledno gibljejo okoli podobnih vrednosti 
največje stopnje in imajo primerljivo porazdelitev stopenj, kar potrjuje stabilnost opaženih strukturnih lastnosti.

\medskip

Ker smo pri velikih drevesih uporabili iste vrednosti parametra $\alpha$ kot pri majhnih, lahko rezultate obeh \mbox{pristopov}
med seboj neposredno primerjamo in tako dobimo celovit vpogled v vpliv $\alpha$ in razvejanosti na \mbox{Somborjev} indeks.

\subsection{Povzetek ugotovitev}

Na podlagi analize majhnih in velikih grafov smo ugotovili, da z naraščanjem vrednosti parametra
$\alpha \in (0,1)$ vrednost splošnega Somborjevega indeksa monotono narašča, ne glede na število vozlišč. Ta lastnost se izkaže kot stabilna tako pri majhnih kot pri velikih drevesih.

\medskip

Vpliv največje stopnje razvejanosti $\Delta$ se pri tem izkaže kot ključen. Pri majhnih grafih je globalni \mbox{maksimum} Somborjevega indeksa dosežen pri drevesih
z največjo možno stopnjo razvejanosti, saj vozlišče z največ povezavami prispeva največji delež k vsoti v
definiciji indeksa. Analiza velikih grafov pokaže, da ta lastnost ostaja ohranjena tudi za večje vrednosti $n$,
saj zvezdno drevo z $\Delta = n-1$ še vedno dosega največjo možno vrednost indeksa.

\medskip

Kljub temu pa rezultati pokažejo pomembne razlike v strukturi optimalnih dreves pri omejeni največji stopnji
$\Delta$. Pri večjih grafih se optimalne konfiguracije za fiksno $\Delta$ porazdelijo na več vozlišč z visoko
stopnjo, kar kaže, da vrednost indeksa ni določena zgolj z enim lokalnim ekstremom, temveč z globalno
porazdelitvijo stopenj v grafu.

\medskip

Uporaba različnih vrednosti parametra $\alpha$, vključno z vrednostmi v bližini $0$, $0.5$ in $1$, je omogočila
celovito analizo vpliva parametra na obnašanje indeksa. S tem smo pokazali, da parameter $\alpha$ ne vpliva
zgolj na absolutno velikost Somborjevega indeksa, temveč tudi na relativni pomen posameznih strukturnih
lastnosti grafa, zlasti razporeditve stopenj vozlišč.


\section*{Viri}

\begin{itemize}
\item Ahmad S., Farooq R., Das K. C.: \emph{General Sombor Index}. MATCH Commun. Math. Comput. Chem. 94 (2020), 825–853.
\end{itemize}

\end{document}