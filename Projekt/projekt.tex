\documentclass{article}

\usepackage[slovene]{babel}
\usepackage{amsmath}
\usepackage{amsfonts} 
\usepackage{amssymb}
\usepackage{mathrsfs}
\usepackage{geometry}
\usepackage{dsfont}
\usepackage{graphicx}
\usepackage{float}
\usepackage[T1]{fontenc}
\usepackage[utf8]{inputenc}

\geometry{a4paper, top=2.5cm, bottom=2.5cm, left=2.0cm, right=2.0cm}
\newtheorem{definition}{Definicija}[section] 

\begin{document}

\thispagestyle{empty}

\begin{center}
    \normalfont\normalsize
    \textsc{Fakulteta za matematiko in fiziko\\Univerza v Ljubljani}\\[25pt]
    \rule{\linewidth}{0.5pt}\\[20pt]
    {\huge \textbf{Splošni Somborjev indeks}}\\[12pt]
    \rule{\linewidth}{2pt}\\[18pt]
    {\large \textsc{Poročilo projekta}}
\end{center}

\vfill
\begin{center}
    \large Pecoraro Eliza Katarina, Žefran Kaja\\[18pt]
    \normalsize december 2025
\end{center}

\newpage

\section{Analiza grafov}

V tem poglavju analiziramo obnašanje Somborjevega indeksa pri različnih vrednostih
parametra $\alpha$ na grafih različnih velikosti. Celotna raziskava je omejena na
drevesa, saj ta predstavljajo osnovni, a hkrati dovolj raznolik razred grafov, pri
katerem je mogoč popoln pregled vseh neizomorfnih struktur. Z analizo želimo raziskati
vpliv parametra $\alpha \in (0,1)$ na vrednost Somborjevega indeksa ter preučiti vpliv
stopnje razvejanosti in števila vozlišč.

\medskip

V nadaljevanju najprej obravnavamo majhne grafe, kjer je zaradi omejenega števila
vozlišč mogoč izčrpen računski pregled vseh neizomorfnih dreves, nato pa analizo
razširimo še na večje grafe.


\subsection{Majhni grafi}

Pri analizi majhnih grafov, ki vsebujejo vse do 10 vozlišč, opazimo izrazit in stabilen trend
v obnašanju \mbox{Somborjevega indeksa}. Za vse uporabljene vrednosti parametra
$\alpha \in (0,1)$ se izkaže, da je vrednost Somborjevega indeksa največja pri grafih z
največjo možno maksimalno stopnjo $\Delta$, glede na število vozlišč. To pomeni, da imajo
vozlišča z največjim številom povezav oziroma z največjo stopnjo ključno vlogo pri velikosti
indeksa, saj bistveno zaznamujejo celotno strukturo grafa.

\medskip
V numerični analizi majhnih grafov smo uporabili vrednosti parametra
\[
\alpha \in \{0.05, 0.45, 0.5, 0.55, 0.95\},
\]
ki ležijo znotraj intervala $(0,1)$. Izbrane vrednosti omogočajo opazovanje obnašanja
indeksa v bližini mejnih vrednosti $0$ in $1$ ter v okolici srednje vrednosti $0.5$.
Rezultati so pokazali, da večja kot je vrednost parametra $\alpha$, večja je tudi
vrednost Somborjevega indeksa, ne glede na število vozlišč ali stopnjo razvejanosti grafa.

\medskip

Za vsako fiksno število vozlišč $n$ smo izvedli popoln računski pregled vseh neizomorfnih
dreves. Za vsak graf smo najprej izračunali največjo stopnjo $\Delta$, nato pa drevesa
razvrstili v razrede glede na to vrednost. Znotraj vsakega razreda smo izračunali
Somborjev indeks, definiran z
\[
SO_\alpha(G) = \sum_{uv \in E(G)} (d_u^2 + d_v^2)^\alpha,
\]
kjer $d_u$ in $d_v$ označujeta stopnji krajišč roba $uv$.
Na ta način smo za vsak $\Delta$ določili drevo, pri katerem Somborjev indeks doseže
največjo vrednost. Poleg tega smo za vsako izbrano vrednost parametra $\alpha$
določili tudi globalni maksimum Somborjevega indeksa med vsemi drevesi z $n$ vozlišči.

\medskip

Kot reprezentativen primer si oglejmo primer $n = 7$.
Rezultati pokažejo, da Somborjev indeks za vse obravnavane vrednosti parametra $\alpha$
monotono narašča z naraščanjem maksimalne stopnje $\Delta$.
Globalni maksimum je v vseh primerih dosežen pri drevesu z $\Delta = 6$,
kar ustreza zvezdastemu drevesu.
To drevo vsebuje eno vozlišče stopnje $6$, ki je povezano z vsemi preostalimi vozlišči,
kar povzroči največji možni prispevek k vrednosti indeksa.

Opazimo tudi, da se z večanjem parametra $\alpha$ razlike med vrednostmi
Somborjevega indeksa pri različnih vrednostih $\Delta$ še povečujejo.
To pomeni, da večje vrednosti $\alpha$ dodatno poudarijo prispevek robov,
ki povezujejo vozlišča z visokimi stopnjami, kar pride posebej do izraza
pri zvezdastih drevesih.

\medskip

Na podlagi izčrpne računske analize majhnih grafov lahko sklenemo, da za vsako
obravnavano število vozlišč $n \leq 10$ in za vsak parameter $\alpha \in (0,1)$
$\alpha$-Somborjev indeks na razredu dreves doseže največjo vrednost pri grafih z
največjo možno maksimalno stopnjo $\Delta = n-1$.
Poleg tega se je pokazalo, da število povezav vpliva na absolutno vrednost indeksa,
vendar ne spremeni njegovega kvalitativnega obnašanja glede na parameter $\alpha$.
Ti rezultati predstavljajo trdno osnovo za nadaljnjo analizo večjih grafov.


\newpage

\subsection{Veliki grafi}

Pri analizi velikih grafov, so bila opažanja podobna tistim, ki smo jih 
naredili pri majhnih grafih, vendar z nekaj pomembnimi razlikami. Ugotovili smo, da so večji grafi imeli večjo 
absolutno vrednost Somborjevega indeksa, vendar so razmerja med različnimi vrednostmi parametra $ \alpha \in (0, 1) $ 
in stopnjo razvejanosti ostala precej podobna.

Pri velikih grafih se je pokazalo, da je največja stopnja razvejanosti $ \Delta $, ki je tudi ključna za analizo 
majhnih grafov, še vedno imela največji vpliv na vrednost Somborjevega indeksa. Ne glede na število vozlišč in 
razvejanost grafa je bil indeks pri večjih vrednostih parametra $ \alpha $ večji. To potrjuje, da je vpliv parametra 
$ \alpha $ dosleden tudi pri večjih in bolj zapletenih grafih.

Poleg tega smo opazili, da večje število vozlišč in večja povezanost med njimi (večje število povezav) niso pomembno spremenili 
relativnega obnašanja indeksa v različnih vrednostih $ \alpha $. To nakazuje, da so opažene lastnosti Somborjevega indeksa 
stabilne, ne glede na velikost grafa. Kljub temu, da večji grafi vplivajo na absolutne vrednosti indeksa, so še vedno ohranjene 
ključne značilnosti, ki smo jih opazili pri manjših grafih.

\subsection{Povzetek ugotovitev}

Na podlagi analize majhnih in velikih grafov smo ugotovili, da večja vrednost parametra $ \alpha \in (0, 1) $ 
povzroči večji Somborjev indeks, ne glede na število vozlišč ali stopnjo razvejanosti v grafu. Ključen dejavnik pri 
obnašanju indeksa je bila največja stopnja razvejanosti $ \Delta $, saj vozlišča z največ povezavami (največja stopnja) 
povzročajo največji vpliv na vrednost indeksa. Uporaba različnih vrednosti parametra $ \alpha $, vključno z vrednostmi v 
bližini 0, 0.5 in 1, je omogočila natančno analizo, kako te vrednosti vplivajo na rezultat in omogočile razumevanje obnašanja 
indeksa v različnih okoljih.


\end{document}