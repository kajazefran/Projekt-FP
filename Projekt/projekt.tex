\documentclass{article}

\usepackage[slovene]{babel}
\usepackage{amsmath}
\usepackage{amsfonts} 
\usepackage{amssymb}
\usepackage{mathrsfs}
\usepackage{geometry}
\usepackage{dsfont}
\usepackage{graphicx}
\usepackage{float}
\usepackage[T1]{fontenc}
\usepackage[utf8]{inputenc}

\geometry{a4paper, top=2.5cm, bottom=2.5cm, left=2.0cm, right=2.0cm}
\newtheorem{definition}{Definicija}[section] 

\begin{document}

\thispagestyle{empty}

\begin{center}
    \normalfont\normalsize
    \textsc{Fakulteta za matematiko in fiziko\\Univerza v Ljubljani}\\[25pt]
    \rule{\linewidth}{0.5pt}\\[20pt]
    {\huge \textbf{Splošni Somborjev indeks}}\\[12pt]
    \rule{\linewidth}{2pt}\\[18pt]
    {\large \textsc{Poročilo projekta}}
\end{center}

\vfill
\begin{center}
    \large Pecoraro Eliza Katarina, Žefran Kaja\\[18pt]
    \normalsize december 2025
\end{center}

\newpage

\section{Analiza grafov}

V tem poglavju analiziramo obnašanje Somborjevega indeksa pri različnih vrednostih
parametra $\alpha$ na grafih različnih velikosti. Celotna raziskava je omejena na
drevesa, saj ta predstavljajo osnovni, a hkrati dovolj raznolik razred grafov, pri
katerem je mogoč popoln pregled vseh neizomorfnih struktur. Z analizo želimo raziskati
vpliv parametra $\alpha \in (0,1)$ na vrednost Somborjevega indeksa ter preučiti vpliv
stopnje razvejanosti in števila vozlišč.

\medskip

V nadaljevanju najprej obravnavamo majhne grafe, kjer je zaradi omejenega števila
vozlišč mogoč izčrpen računski pregled vseh neizomorfnih dreves, nato pa analizo
razširimo še na večje grafe.


\subsection{Majhni grafi}

Pri analizi majhnih grafov, ki vsebujejo vse do 10 vozlišč, opazimo izrazit in stabilen trend
v obnašanju \mbox{Somborjevega indeksa}. Za vse uporabljene vrednosti parametra
$\alpha \in (0,1)$ se izkaže, da je vrednost Somborjevega indeksa največja pri grafih z
največjo možno maksimalno stopnjo $\Delta$, glede na število vozlišč. To pomeni, da imajo
vozlišča z največjim številom povezav oziroma z največjo stopnjo ključno vlogo pri velikosti
indeksa, saj bistveno zaznamujejo celotno strukturo grafa.

\medskip
V numerični analizi majhnih grafov smo uporabili vrednosti parametra
\[
\alpha \in \{0.05, 0.45, 0.5, 0.55, 0.95\},
\]
ki ležijo znotraj intervala $(0,1)$. Izbrane vrednosti omogočajo opazovanje obnašanja
indeksa v bližini mejnih vrednosti $0$ in $1$ ter v okolici srednje vrednosti $0.5$.
Rezultati so pokazali, da večja kot je vrednost parametra $\alpha$, večja je tudi
vrednost Somborjevega indeksa, ne glede na število vozlišč ali stopnjo razvejanosti grafa.

\medskip

Za vsako fiksno število vozlišč $n$ smo izvedli popoln računski pregled vseh neizomorfnih
dreves. Za vsak graf smo najprej izračunali največjo stopnjo $\Delta$, nato pa drevesa
razvrstili v razrede glede na to vrednost. Znotraj vsakega razreda smo izračunali
Somborjev indeks, definiran z
\[
SO_\alpha(G) = \sum_{uv \in E(G)} (d_u^2 + d_v^2)^\alpha,
\]
kjer $d_u$ in $d_v$ označujeta stopnji krajišč roba $uv$.
Na ta način smo za vsak $\Delta$ določili drevo, pri katerem Somborjev indeks doseže
največjo vrednost. Poleg tega smo za vsako izbrano vrednost parametra $\alpha$
določili tudi globalni maksimum Somborjevega indeksa med vsemi drevesi z $n$ vozlišči.

\medskip

Kot reprezentativen primer si oglejmo primer $n = 7$.
Rezultati pokažejo, da Somborjev indeks za vse obravnavane vrednosti parametra $\alpha$
monotono narašča z naraščanjem maksimalne stopnje $\Delta$.
Globalni maksimum je v vseh primerih dosežen pri drevesu z $\Delta = 6$,
kar ustreza zvezdastemu drevesu.
To drevo vsebuje eno vozlišče stopnje $6$, ki je povezano z vsemi preostalimi vozlišči,
kar povzroči največji možni prispevek k vrednosti indeksa.

Opazimo tudi, da se z večanjem parametra $\alpha$ razlike med vrednostmi
Somborjevega indeksa pri različnih vrednostih $\Delta$ še povečujejo.
To pomeni, da večje vrednosti $\alpha$ dodatno poudarijo prispevek robov,
ki povezujejo vozlišča z visokimi stopnjami, kar pride posebej do izraza
pri zvezdastih drevesih.

\medskip

Na podlagi izčrpne računske analize majhnih grafov lahko sklenemo, da za vsako
obravnavano število vozlišč $n \leq 10$ in za vsak parameter $\alpha \in (0,1)$
$\alpha$-Somborjev indeks na razredu dreves doseže največjo vrednost pri grafih z
največjo možno maksimalno stopnjo $\Delta = n-1$.
Poleg tega se je pokazalo, da število povezav vpliva na absolutno vrednost indeksa,
vendar ne spremeni njegovega kvalitativnega obnašanja glede na parameter $\alpha$.
Ti rezultati predstavljajo trdno osnovo za nadaljnjo analizo večjih grafov.


\newpage

\subsection{Veliki grafi}

Pri analizi velikih grafov se osnovne lastnosti Somborjevega indeksa v določenem smislu ohranijo, vendar se
v primerjavi z majhnimi grafi pojavijo pomembne strukturne razlike. Absolutne vrednosti indeksa z naraščanjem
števila vozlišč sicer naraščajo, vendar grafi z največjo možno stopnjo razvejanosti $\Delta$ ne zagotavljajo več
maksimalne vrednosti Somborjevega indeksa.

\medskip

Pri velikih drevesih se izkaže, da ekstremna koncentracija povezav v enem vozlišču postane manj ugodna.
Čeprav ima vozlišče z zelo veliko stopnjo pomemben prispevek k posameznim členom vsote v definiciji
Somborjevega indeksa, se pri takšnih strukturah zmanjša število robov, ki povezujejo vozlišča z
zmerno velikimi stopnjami. Posledično celotni prispevek vseh robov ni več optimalen.

\medskip

Analiza kaže, da se maksimum Somborjevega indeksa pri velikih grafih doseže pri drevesih z vmesno stopnjo
razvejanosti, kjer so stopnje vozlišč bolj enakomerno porazdeljene. Takšna struktura omogoča, da večje
število robov prispeva k vsoti z dovolj velikimi vrednostmi izraza
\[
(d_u^2 + d_v^2)^\alpha,
\]
kar je pri večjih vrednostih parametra $\alpha$ ugodneje kot posamezni zelo veliki prispevki, ki jih ustvarja
eno samo močno razvejano vozlišče.

\medskip

Poleg tega se z naraščanjem števila vozlišč optimalna vrednost največje stopnje $\Delta$ postopoma zmanjšuje
glede na velikost grafa. To kaže na obstoj prehoda v optimalni strukturi dreves: medtem ko so pri majhnih
grafih optimalna drevesa z maksimalno razvejanostjo, pri velikih grafih optimalna struktura teži k
zmerni razvejanosti in bolj uravnoteženi porazdelitvi stopenj.

\medskip

Vpliv parametra $\alpha \in (0,1)$ ostaja tudi pri velikih grafih bistven. S povečevanjem vrednosti parametra
$\alpha$ se vrednost Somborjevega indeksa povečuje, vendar se hkrati okrepi pomen porazdelitve stopenj
vozlišč v celotnem grafu. To potrjuje, da pri velikih grafih vrednost indeksa ni določena zgolj z lokalnimi
ekstremi, temveč predvsem z globalno strukturo drevesa.

\subsection{Povzetek ugotovitev}

Na podlagi analize majhnih in velikih grafov smo ugotovili, da z naraščanjem vrednosti parametra
$\alpha \in (0,1)$ vrednost $\alpha$-Somborjevega indeksa monotono narašča, ne glede na velikost grafa.
Ta lastnost se izkaže kot stabilna tako pri majhnih kot pri velikih drevesih.

\medskip

Vpliv največje stopnje razvejanosti $\Delta$ se pri tem izkaže kot ključen, vendar se njegova vloga razlikuje
glede na velikost grafa. Pri majhnih grafih je maksimum Somborjevega indeksa dosežen pri drevesih z največjo
možno stopnjo razvejanosti, saj vozlišče z največ povezavami prispeva največji delež k vsoti v definiciji indeksa.
Pri večjih grafih pa se optimalna vrednost $\Delta$ premakne proti vmesnim vrednostim, kar kaže, da indeks ni
odvisen zgolj od lokalnih ekstremov, temveč predvsem od globalne porazdelitve stopenj v grafu.

\medskip

Uporaba različnih vrednosti parametra $\alpha$, vključno z vrednostmi v bližini $0$, $0{,}5$ in $1$, je omogočila
celovito analizo vpliva parametra na obnašanje indeksa. S tem smo pokazali, da parameter $\alpha$ ne vpliva zgolj
na absolutno velikost Somborjevega indeksa, temveč tudi na relativni pomen posameznih strukturnih lastnosti grafa,
zlasti razporeditve stopenj vozlišč.

\end{document}